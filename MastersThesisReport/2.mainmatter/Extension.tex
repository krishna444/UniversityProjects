\chapter{Limitations \& Future Work}

The image alignment methods implemented in the current version of stitching are giving very good result for different intensity, scaling or rotated images. Although the methods are optimized to get faster and better result, still for very high resolution images, the process is bulky. So, an extension of this project could be to carry out research on obtaining more optimized algorithms to work for very high resolution images. An X-ray image generally consists of a lot of background region which consists of very little or no information for image stitching. So, we can implement some technique that selects bones and muscles and discard the other areas in the image.\\

\noindent There still needs to work more on the blending part. For different intensity images, if the images are rotated, the current blending creates a seam on the overlapped region. The problem is because of unassigned pixels created when image is rotated. The current version employs the filling of unassigned pixels with pixels of other image which works perfect for images with same intensity. Because of intensity difference, the unassigned pixels get intensity which is different from their actual pixels and we see visible line after blending. The extension could be to use some transformation method to get the accurate pixel intensity to be filled in the unassigned pixels.

