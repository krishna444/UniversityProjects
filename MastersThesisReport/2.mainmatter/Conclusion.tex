\chapter{Conclusion}
This thesis evaluates image analysis methods that can be used for stitching of medical X-ray images. Generally, image stitching process includes several steps. If we do not improve or optimize the methods, the process takes a lot of time that makes it inappropriate for real-time or user interactive applications. I have analyzed, evaluated and optimized the methods based on computational complexity and accuracy to get faster stitching result.\\

\noindent Image alignment using feature points are faster and always guaranteed to give best result while other methods such as pixel based alignment, area or pattern based alignment are slower and not guaranteed to provide best result. I have analyzed and evaluated three most popular feature extractors: \emph{Harris}, \emph{SIFT} and \emph{SURF}. Harris is good for faster detection of corners in the image, but it can not give accurate corners if the images to be stitched have different intensity or orientation. So, for feature extractors, either SIFT or SURF methods can be employed. For feature matching, although, it gives accurate result, SIFT is suffering from its high computational complexity, so, SURF is better option because it is faster than SIFT. The exhaustive \emph{nearest neighborhood} (kNN) method is modified to \emph{approximate nearest neighborhood} (ANN) to get faster matching while sacrificing minor matching inaccuracies. Since, nearest neighborhood based methods gives just nearest key-points as a match which is not always a true match. The application of accuracy tests such as \emph{Ratio Test} and \emph{Symmetry Test} are quite effective to remove those false matches. The use of SURF features matching and approximate nearest neighborhood methods are intended to make the stitching process faster sacrificing some minor inaccuracy in matching resulting from those methods.\\ 

\noindent The transformation model is used to create the composite image. The homography matrix is suitable transformation model for medical images. The \emph{RANSAC} method is preferable for robust estimation of homography because it works quite effectively even if we have a lot of false matches.\\

\noindent I have evaluated two popular blending techniques: \emph{alpha blending} and \emph{pyramid blending}. Although pyramid blending is an effective method for blending, it is computationally expensive and loses image information. So, alpha blending is chosen for blending which is faster and gives more accurate result. The alpha blending can be modified to work on complex alignments which occurs if the images are rotated and/or are taken in perspective view. The modified alpha blending use blending masks to assign the weights of the pixels of contributing images on the blended region. The method is effective because it successfully removes the seams and discontinuities on the composite image.

 


